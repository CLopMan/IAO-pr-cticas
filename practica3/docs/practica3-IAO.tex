%----------
%   IMPORTANTE
%----------

% Si nunca has utilizado LaTeX es conveniente que aprendas una serie de conceptos básicos antes de utilizar esta plantilla. Te aconsejamos que leas previamente algún tutorial (puedes encontar muchos en Internet).

% Esta plantilla está basada en las recomendaciones de la guía "Trabajo fin de Grado: Escribir el TFG", que encontrarás en http://uc3m.libguides.com/TFG/escribir
% contiene recomendaciones de la Biblioteca basadas principalmente en estilos APA e IEEE, pero debes seguir siempre las orientaciones de tu Tutor de TFG y la normativa de TFG para tu titulación.

% Encontrarás un ejemplo de TFG realizado con esta misma plantilla en la carpeta "_ejemplo_TFG_2019". Consúltalo porque contiene ejemplos útiles para incorporar tablas, figuras, listados de código, bibliografía, etc.


%----------
%    CONFIGURACIÓN DEL DOCUMENTO
%----------

% Definimos las características del documento y añadimos una serie de paquetes (\usepackage{package}) que agregan funcionalidades a LaTeX.

\documentclass[12pt]{report} %fuente a 12pt

% MÁRGENES: 2,5 cm sup. e inf.; 3 cm izdo. y dcho.
\usepackage[
a4paper,
vmargin=2.5cm,
hmargin=3cm
]{geometry}

% INTERLINEADO: Estrecho (6 ptos./interlineado 1,15) o Moderado (6 ptos./interlineado 1,5)
\renewcommand{\baselinestretch}{1.15}
\parskip=6pt

\providecommand{\tightlist}{%
  \setlength{\itemsep}{0pt}\setlength{\parskip}{0pt}}

% DEFINICIÓN DE COLORES para portada y listados de código
\usepackage[table]{xcolor}
\definecolor{azulUC3M}{RGB}{0,0,102}
\definecolor{gray97}{gray}{.97}
\definecolor{gray75}{gray}{.75}
\definecolor{gray45}{gray}{.45}

\usepackage{tikz}
% Soporte para GENERAR PDF/A --es importante de cara a su inclusión en e-Archivo porque es el formato óptimo de preservación y a la generación de metadatos, tal y como se describe en http://uc3m.libguides.com/ld.php?content_id=31389625. En la carpeta incluímos el archivo plantilla_tfg_2017.xmpdata en el que puedes incluir los metadatos que se incorporarán al archivo PDF cuando lo compiles. Ese archivo debe llamarse igual que tu archivo .tex. Puedes ver un ejemplo en esta misma carpeta.
\usepackage[a-1b]{pdfx}

% ENLACES
\usepackage{hyperref}
\hypersetup{colorlinks=true,
    linkcolor=black, % enlaces a partes del documento (p.e. índice) en color negro
    citecolor=black,
    urlcolor=blue} % enlaces a recursos fuera del documento en azul

% EXPRESIONES MATEMATICAS
\usepackage{amsmath,amssymb,amsfonts,amsthm}
\usepackage{booktabs}

\usepackage{txfonts}
\usepackage[T1]{fontenc}
\usepackage[utf8]{inputenc}

\usepackage[spanish, es-tabla]{babel}
\usepackage[babel, spanish=spanish]{csquotes}
\AtBeginEnvironment{quote}{\small}

% diseño de PIE DE PÁGINA
\usepackage{fancyhdr}
\pagestyle{fancy}
\fancyhf{}
\renewcommand{\headrulewidth}{0pt}
\rfoot{\thepage}
\fancypagestyle{plain}{\pagestyle{fancy}}

% DISEÑO DE LOS TÍTULOS de las partes del trabajo (capítulos y epígrafes o subcapítulos)
\usepackage{titlesec}
\usepackage{titletoc}
\titleformat{\chapter}[block]
{\large\bfseries\filcenter}
{\thechapter.}
{5pt}
{\MakeUppercase}
{}
\titlespacing{\chapter}{0pt}{0pt}{*3}
\titlecontents{chapter}
[0pt]
{}
{\contentsmargin{0pt}\thecontentslabel.\enspace\uppercase}
{\contentsmargin{0pt}\uppercase}
{\titlerule*[.7pc]{.}\contentspage}

\titleformat{\section}
{\bfseries}
{\thesection.}
{5pt}
{}
\titlecontents{section}
[5pt]
{}
{\contentsmargin{0pt}\thecontentslabel.\enspace}
{\contentsmargin{0pt}}
{\titlerule*[.7pc]{.}\contentspage}

\titleformat{\subsection}
{\normalsize\bfseries}
{\thesubsection.}
{5pt}
{}
\titlecontents{subsection}
[10pt]
{}
{\contentsmargin{0pt}
    \thecontentslabel.\enspace}
{\contentsmargin{0pt}}
{\titlerule*[.7pc]{.}\contentspage}


% DISEÑO DE TABLAS. Puedes elegir entre el estilo para ingeniería o para ciencias sociales y humanidades. Por defecto, está activado el estilo de ingeniería. Si deseas utilizar el otro, comenta las líneas del diseño de ingeniería y descomenta las del diseño de ciencias sociales y humanidades
\usepackage{multirow} % permite combinar celdas
\usepackage{caption} % para personalizar el título de tablas y figuras
\usepackage{floatrow} % utilizamos este paquete y sus macros \ttabbox y \ffigbox para alinear los nombres de tablas y figuras de acuerdo con el estilo definido. Para su uso ver archivo de ejemplo
\usepackage{array} % con este paquete podemos definir en la siguiente línea un nuevo tipo de columna para tablas: ancho personalizado y contenido centrado
\newcolumntype{P}[1]{>{\centering\arraybackslash}p{#1}}
\DeclareCaptionFormat{upper}{#1#2\uppercase{#3}\par}

% Diseño de tabla para ingeniería
\captionsetup[table]{
    format=upper,
    name=TABLA,
    justification=centering,
    labelsep=period,
    width=.75\linewidth,
    labelfont=small,
    font=small,
}

%Diseño de tabla para ciencias sociales y humanidades
%\captionsetup[table]{
%    justification=raggedright,
%    labelsep=period,
%    labelfont=small,
%    singlelinecheck=false,
%    font={small,bf}
%}


% DISEÑO DE FIGURAS. Puedes elegir entre el estilo para ingeniería o para ciencias sociales y humanidades. Por defecto, está activado el estilo de ingeniería. Si deseas utilizar el otro, comenta las líneas del diseño de ingeniería y descomenta las del diseño de ciencias sociales y humanidades
\usepackage{graphicx}
\graphicspath{{img/}} %ruta a la carpeta de imágenes

% Diseño de figuras para ingeniería
\captionsetup[figure]{
    format=hang,
    name=Fig.,
    singlelinecheck=off,
    labelsep=period,
    labelfont=small,
    font=small
}

% Diseño de figuras para ciencias sociales y humanidades
%\captionsetup[figure]{
%    format=hang,
%    name=Figura,
%    singlelinecheck=off,
%    labelsep=period,
%    labelfont=small,
%    font=small
%}


% NOTAS A PIE DE PÁGINA
\usepackage{chngcntr} %para numeración contínua de las notas al pie
\counterwithout{footnote}{chapter}

% LISTADOS DE CÓDIGO
% soporte y estilo para listados de código. Más información en https://es.wikibooks.org/wiki/Manual_de_LaTeX/Listados_de_código/Listados_con_listings
\usepackage{listings}

% definimos un estilo de listings
\lstdefinestyle{estilo}{ frame=Ltb,
    framerule=0pt,
    aboveskip=0.5cm,
    framextopmargin=3pt,
    framexbottommargin=3pt,
    framexleftmargin=0.4cm,
    framesep=0pt,
    rulesep=.4pt,
    backgroundcolor=\color{gray97},
    rulesepcolor=\color{black},
    %
    basicstyle=\ttfamily\footnotesize,
    keywordstyle=\bfseries,
    stringstyle=\ttfamily,
    showstringspaces = false,
    commentstyle=\color{gray45},
    %
    numbers=left,
    numbersep=15pt,
    numberstyle=\tiny,
    numberfirstline = false,
    breaklines=true,
    xleftmargin=\parindent
}

\captionsetup[lstlisting]{font=small, labelsep=period}
% fijamos el estilo a utilizar
\lstset{style=estilo}
\renewcommand{\lstlistingname}{\uppercase{Código}}


%BIBLIOGRAFÍA - PUEDES ELEGIR ENTRE ESTILO IEEE O APA. POR DEFECTO ESTÁ CONFIGURADO IEEE. SI DESEAS USAR APA, COMENTA LAS LÍNEA DE IEEE Y DESCOMENTA LAS DE APA. Si haces cambios en la configuración de la bibliografía y no obtienes los resultados esperados, es recomendable limpiar los archivos auxiliares y volver a compilar en este orden: COMPILAR-BIBLIOGRAFIA-COMPILAR

% Tienes más información sobre cómo generar bibliografía y CONFIGURAR TU EDITOR DE TEXTO para compilar con biber en http://tex.stackexchange.com/questions/154751/biblatex-with-biber-configuring-my-editor-to-avoid-undefined-citations , https://www.overleaf.com/learn/latex/Bibliography_management_in_LaTeX y en http://www.ctan.org/tex-archive/macros/latex/exptl/biblatex-contrib
% También te recomendamos consultar la guía temática de la Biblioteca sobre citas bibliográficas: http://uc3m.libguides.com/guias_tematicas/citas_bibliograficas/inicio

% CONFIGURACIÓN PARA LA BIBLIOGRAFÍA IEEE
\usepackage[backend=biber, style=ieee, isbn=false,sortcites, maxbibnames=5, minbibnames=1]{biblatex} % Configuración para el estilo de citas de IEEE, recomendado para el área de ingeniería. "maxbibnames" indica que a partir de 5 autores trunque la lista en el primero (minbibnames) y añada "et al." tal y como se utiliza en el estilo IEEE.

%CONFIGURACIÓN PARA LA BIBLIOGRAFÍA APA
%\usepackage[style=apa, backend=biber, natbib=true, hyperref=true, uniquelist=false, sortcites]{biblatex}
%\DeclareLanguageMapping{spanish}{spanish-apa}

% Añadimos las siguientes indicaciones para mejorar la adaptación de los estilos en español
\DefineBibliographyStrings{spanish}{%
    andothers = {et\addabbrvspace al\adddot}
}
\DefineBibliographyStrings{spanish}{
    url = {\adddot\space[En línea]\adddot\space Disponible en:}
}
\DefineBibliographyStrings{spanish}{
    urlseen = {Acceso:}
}
\DefineBibliographyStrings{spanish}{
    pages = {pp\adddot},
    page = {p.\adddot}
}

\addbibresource{references.bib} % llama al archivo bibliografia.bib en el que debería estar la bibliografía utilizada


%-------------
%    DOCUMENTO
%-------------

\begin{document}
\pagenumbering{roman} % Se utilizan cifras romanas en la numeración de las páginas previas al cuerpo del trabajo

%----------
%    PORTADA
%----------
\title{Práctica 1}
\author{Álvaro Guerrero Espinosa (100472294)\\
        César López Mantecón (100472092)\\
        Paula Subías Serrano (100472119)\\
        Irene Subías Serrano (100472108)\\}

\makeatletter
\begin{titlepage}
    \begin{sffamily}
    \color{azulUC3M}
    \begin{center}
        \begin{figure}[H] %incluimos el logotipo de la Universidad
            \makebox[\textwidth][c]{\includegraphics[width=16cm]{Portada_Logo.png}}
        \end{figure}
        \vspace{2.5cm}
        \begin{Large}
            Grado en Ingeniería Informática\\
            \@date\\
            \vspace{2cm}
            \textsl{Inteligencia Artificial en las Organizaciones}\\
            \bigskip
        \end{Large}
        {\Huge ``\@title''}\\
        \vspace*{0.5cm}
        \rule{10.5cm}{0.1mm}\\
        \vspace*{0.9cm}
        {\LARGE\@author}
        \vspace*{1cm}
    \end{center}
    \vfill
    \color{black}
    % si nuestro trabajo se va a publicar con una licencia Creative Commons, incluir estas líneas. Es la opción recomendada.
    \includegraphics[width=4.2cm]{creativecommons.png}\\ %incluimos el logotipo de creativecommons
    Esta obra se encuentra sujeta a la licencia Creative Commons \textbf{Reconocimiento - No Comercial - Sin Obra Derivada}
    \end{sffamily}
\end{titlepage}
\makeatother

\newpage %página en blanco o de cortesía
\thispagestyle{empty}
\mbox{}

%----------
%    ÍNDICES
%----------

%--
% Índice general
%-
\tableofcontents
\thispagestyle{fancy}

\newpage % página en blanco o de cortesía
\thispagestyle{empty}
\mbox{}

%--
% Índice de figuras. Si no se incluyen, comenta las líneas siguientes
%-
 \listoffigures
 \thispagestyle{fancy}

 \newpage % página en blanco o de cortesía
 \thispagestyle{empty}
 \mbox{}

%--
% Índice de tablas. Si no se incluyen, comenta las líneas siguientes
%-
\listoftables
 \thispagestyle{fancy}

 \newpage % página en blanco o de cortesía
 \thispagestyle{empty}
 \mbox{}


%----------
%    TRABAJO
%----------
\clearpage
\pagenumbering{arabic} % numeración con múmeros arábigos para el resto de la publicación

    \chapter{Introducción y objetivo}
    \label{chap:intro}

        En este documento se recoge el desarrollo de la práctica 3 de la
        asignatura \textit{Inteligencia Artificial en las Organizaciones}. El
        objetivo de este proyecto es crear un sistema de puntuación de inmuebles 
        basado en el perfil de un estudiante. Para ello se empleará un sistema
        de lógica borrosa.

        La creación de las reglas del sistema se centrarán en el perfil de un
        estudiante que busca un piso para vivir en solitario durante sus
        estudios. Se ha elegido este perfil por ser del que se tiene más
        información acerca de sus preferencias y tendencias de compra, ya que
        forma parte del colectivo de los investigadores.

        La vivienda en las grandes ciudades españolas presenta un gran problema
        a colectivos vulnerables, entre ellos los
        estudiantes~\cite{viviendas-precio-inaccesible}. Esto, en combinación
        con la escasa educación financiera de los estudiantes
        españoles~\cite{PISA-2022}, hace de un sistema de recomendación
        centrado en este perfil un elemento útil para facilitar a los
        estudiantes la evaluación de un activo inmobiliario para su alquiler
        durante sus años de estudio.

        Adicionalmente, el uso de lógica borrosa presenta claras ventajas dada
        la naturaleza del problema:
        \begin{itemize} \item \textbf{Gestión de incertidumbre y ambigüedad}: a
        través de las funciones de pertenencia para cada etiqueta los sistemas
        basados en reglas borrosas son capaces de manejar datos ambigüos o
        incompletos. Esta es una característica fundamental para el sistema que
        se pretende construir.

            \item \textbf{Tolerancia a rangos y datos imprecisos}: dado que las
            variables están fuertemente ligadas a la opinión subjetiva de los
            usuarios, el uso de lógica borrosa permite que el sistema trabaje
            con valores como ``caro'' o ``lejos'' y rangos de valores que
            permiten representar la imprecisión inherente a valores subjetivos.

            \item \textbf{Flexibilidad}: las reglas de un sistema basado en
            reglas borrosas permiten ofrecer recomendaciones más realistas y
            ajustadas al usuario que un sistema que solo aplica umbrales
            rígidos.

            \item \textbf{Combinación de múltiples criterios}: un sistema
            basado en lógica borrosa puede utilizar varios criterios (precio,
            tamaño, número de habitaciones...) a través del concepto de
            \textit{afinidad} de forma que las recomendaciones sean más
            realistas e informadas.

        \end{itemize}

        Con todo lo anterior es evidente que un modelo con las características
        descritas presenta una gran utilidad para el colectivo de estudiantes.
        Además, el uso de logica borrosa presenta claras ventajas para el
        desarrollo de este sistema.

    \chapter{Desarrollo del sistema}
    \label{chap:desarrollo}

        En este capítulo se describe el proceso de creación del sistema,
        incluyendo la elección de atributos de entrada, sus rangos y funciones
        de pertenencia, las etiquetas de la variable de salida y la definición
        de reglas difusas. 
        
        Todos los pasos de este proceso han venido guiados por el perfil
        elegido del estudiante, caracterizado por un bajo poder adquisitivo y
        la búsqueda de un piso relativamente pequeño, cerca de paradas de
        transporte público y preferiblemente amueblado en el que poder vivir en
        solitario hasta la finalización de sus estudios. Con estas
        características se han elegido un total de 7 atributos de entrada así
        como 16 reglas, que pueden representar de manera precisa las
        preferencias de este estudiante. Aunque se tratará como caso principal
        un estudiante que busque un piso en solitario, también se contempla un
        segundo caso en el que prefiera compartir piso. Esto se traduce en
        algunas reglas adicionales dentro del sistema. Para el desarrollo se va
        a utilizar el operador \textit{Fuzzy Logic Toolbox}~\cite{fuzzy-docs}
        de MATLAB para crear un sistema de Mamdani Type1 FIS. Esta herramienta
        permite la selección de variables de entrada y salida así como la
        definición de reglas difusas de manera sencilla y visual.

    \section{Variables de entrada}

        Las variables de entrada seleccionadas para el sistema son una
        combinación de las variables propuestas en el enunciado y otras nuevas
        que permiten representar elementos importantes en las preferencias del
        perfil seleccionado. Para definir sus rangos se ha realizado una
        búsqueda a través de \href{https://www.idealista.com/}{Idealista.com}
        con el fin de determinar los valores máximos y mínimos que alcanzan. En
        algunos casos ha sido necesario tomar un rango más amplio para que la
        función de pertenencia de los conjuntos cubriese el caso del valor 0.

        Los conjuntos borrosos y sus funciones de pertenencia han sido
        definidos según las características de cada uno, teniendo el cuenta el
        rango así como los valores medios de cada variable. Aún así, se ha
        asegurado en todos los casos que el mínimo número de etiquetas borrosas del
        atributo sea 3, con excepción de las variables \textit{booleanas}.

        Las variables utilizadas finalmente son las siguientes: 
        \begin{itemize}
            \item \textbf{Precio}: El precio de alquiler del inmueble en euros por metro cuadrado.

            \item \textbf{Tamaño}: El tamaño en metros cuadrados del inmueble. 
            
            \item \textbf{Número de habitaciones}: El número de habitaciones de cualquier tipo total en la vivienda. 

            \item \textbf{Altura}: Se refiere al piso en el que se encuentra el inmueble.
            
            \item \textbf{Amueblado}: Variable de entrada booleana que indica si el inmueble está amueblado.
            
            \item \textbf{Ascensor}: Variable booleana que corresponde con la presencia de un ascensor en el complejo al que pertenece la vivienda. 

            \item \textbf{Transporte}: La distancia media en minutos de la vivienda a la parada de transporte público más cercana.
        \end{itemize}

    \section{Variables de salida}
    En este apartado se va a describir como se ha definido la variable objetivo "Recomendacion". Esta variable contiene cinco etiquetas borrosas: 
    "Nada recomendable", "Poco recomendable", "Medianamente recomendable", "Recomendable" y "Muy recomendable". Estas represenrtan los niveles de 
    recomendación del piso y serán las que se usarán en las reglas difusas. Aun así la salida que se le otorgará al usuario será un número entre 
    0 y uno que indica el grado de recomendación, siendo 1 lo más recomendado y 0 lo menos. Esto se debe a que aunque el sistema funciona con 
    etiquetas borrosas, el resultado pasa por un proceso de desborrisificacion antes de ser otrogado al usuario. Las 5 funciones de pertenencia 
    tienen forma trapezoidal ya que se ha considerado que cada una de ellas tiene un rango en el que se asegura que son la etiqueta más 
    representativa y por lo tanto un grado de pertenencia 1. Tambien tienen una distribución equitativa a lo largo de la variable ya que se ha 
    decidido que cada una de las etiquetas deben cubrir un rango igual al resto.

    \section{Definición de reglas difusas}

    En este apartado se describen las reglas establecidas para el sistema, que
    han sido definidas a partir del perfil seleccionado. Cabe recalcar que
    aunque se comenzó con 17 reglas, estas han sido modificadas y reducidas a
    16 tras varias ejecuciones para poder conseguir un modelo preciso. Estos
    cambios sobretodo incluyen modificaciones de las etiquetas de salida,
    debido a que el primer modelo tendía en exceso a dar resultados intermedios,
    incluso para ejemplos extremos de viviendas.

    - regla 1,2,3,4
    Las primeras reglas tienen peso máximo y buscan clasificar las viviendas según su precio 
    y tamaño. Estas reglas beneficiarán aquellas viviendas con un menor precio y tamaño, siendo 
    esta decisión justificada por el poco poder asquisitivo y la búsqueda de un piso para vivir solo
    del perfil del estudiante. Además, se considera que estos dos atributos son los que más peso 
    pueden tener sobre la decisión final de elegir el inmueble, lo cual explica el gran peso de 
    estas reglas.

    - regla 5,6,7,16
    Estas reglas tienen relación con el estado de amueblación de la vivienda, considerandose que el 
    estudiante promedio no puede permitirse amueblar la vivienda para uno o dos años. 
    Esta caracteristica tiene relación directa con el precio final de vivir en el lugar elegido, 
    además del coste de tiempo extra que conlleva. Por lo tanto, estas reglas contemplan casos en 
    los que la vivienda no esté amueblada en relación con su precio así como su tamaño. Cabe recalcar
    que la regla 16 se añadió posteriormente para cubrir esta última relación, considerandose que el 
    tamaño aumenta de manera exponencial el coste de necesitar amueblar la vivienda. Así, la mayoría 
    de estas reglas cuentan con el peso máximo, con la excepción de los pisos baratos no amueblados, ya que un
    estudiante podría plantearse una vivienda no amueblada si el precio adicional que esto conlleva 
    se ve balanceado por el bajo precio del alquiler.

    - regla 8,9,10
    Estas reglas buscan reflejar la opción de compartir piso, pero con un peso mucho menor ya que esto 
    es el perfil secundario considerado. La regla 8 en particular cuenta con recomendación negativa, 
    razonando que si una vivienda es pequeña y tiene muchas habitaciones estas serán prácticamente 
    inhabitables. Así, solo se considera de manera positiva aquellos inmuebles de tamaño grande o mediano
    con un número mayor de habitaciones.

    - regla 11
    Caso eliminativo de un piso muy alto si este no tiene ascensor. La altura empieza a 
    considerarse "alta" si la vivienda se encuentra a partir del piso sexto, lo que sin ascensor 
    conlleva un esfuerzo físico diario que se considera excesivo para cualquier persona promedio. 
    Al ser eliminativa cuenta con peso máximo.

    - regla 12,13,14
    Son reglas en relación con el transporte público. Aunque tienen un peso algo menor al de otras
    reglas, se considera que aquellos inmuebles cercanos a transporte público deberían ser mucho 
    más recomendados que los que se encuentran excesivamente lejos. Esta regla viene justificada
    por la importancia del transporte público para el día a día de un estudiante, que suele no contar 
    con un medio de transporte propio.

    - Regla 15
    Se trata de una regla que combina tres atributos distintos y permite que un piso de precio más 
    alto en muy buenas condiciones de posicionamiento y tamaño pueda ser más recomendado. Esta regla busca 
    identificar los pisos que son estudios, siendo estos una muy buena opción para estudiantes aún con 
    un precio por metro cuadrado más alto de lo habitual. 

    \chapter{Listado de propiedades}
    \label{chap:propiedades}

    \begin{table}[H]
        \center
        \begin{tabular}{|c|cccc|}
            \hline
            \textbf{Característica} & \href{https://www.idealista.com/inmueble/106565852/}{Propiedad 1} & \href{https://www.idealista.com/inmueble/97146418/}{Propiedad 2} & \href{https://www.idealista.com/inmueble/104278344/}{Propiedad 3} & \href{https://www.idealista.com/inmueble/106562120/}{Propiedad 4} \\
            \hline
            \hline
            \textbf{Precio (€$/m^2$)}                  & 13.51 & 13.04 & 12.17 & 19.05 \\
            \textbf{Tamaño ($m^2$)}                    & 111   & 92    & 76    & 63    \\
            \textbf{Nº habitaciones}                   & 3     & 2     & 1     & 2     \\
            \textbf{Altura}                            & 2º    & 1º    & 4º    & 1º    \\
            \textbf{Amueblada}                         & 1     & 1     & 1     & 1     \\
            \textbf{Con ascensor}                      & 1     & 1     & 1     & 0     \\
            \textbf{Tiempo a transporte público (min)} & 5     & 10    & 3     & 8     \\
            \hline
        \end{tabular}
        \caption{Listado de propiedades evaluadas - 1}
    \end{table}
    \begin{table}[H]
        \center
        \begin{tabular}{|c|ccc|}
            \hline
            \textbf{Característica} & \href{https://www.idealista.com/inmueble/106559777/}{Propiedad 5} & \href{https://www.idealista.com/inmueble/106224752/}{Propiedad 6} & \href{https://www.idealista.com/inmueble/106137531/}{Propiedad 7} \\
            \hline
            \hline
            \textbf{Precio (€$/m^2$)}                  & 21.57 & 16.84 & 15  \\
            \textbf{Tamaño ($m^2$)}                    & 102   & 95    & 160 \\
            \textbf{Nº habitaciones}                   & 2     & 4     & 4   \\
            \textbf{Altura}                            & 1º    & 2º    & 0º  \\
            \textbf{Amueblada}                         & 1     & 1     & 1   \\
            \textbf{Con ascensor}                      & 1     & 1     & 1   \\
            \textbf{Tiempo a transporte público (min)} & 4     & 1     & 4   \\
            \hline
        \end{tabular}
        \caption{Listado de propiedades evaluadas - 2}
    \end{table}
    \begin{table}[H]
        \center
        \begin{tabular}{|c|ccc|}
            \hline
            \textbf{Característica} & \href{https://www.idealista.com/inmueble/106330042/}{Propiedad 8} & \href{https://www.idealista.com/inmueble/106355273/}{Propiedad 9} & \href{https://www.idealista.com/inmueble/106107441/}{Propiedad 10} \\
            \hline
            \hline
            \textbf{Precio (€$/m^2$)}                  & 18.65 & 14.47 & 10.54 \\
            \textbf{Tamaño ($m^2$)}                    & 48    & 76    & 140   \\
            \textbf{Nº habitaciones}                   & 1     & 2     & 4     \\
            \textbf{Altura}                            & 4º    & 1º    & 1º    \\
            \textbf{Amueblada}                         & 1     & 0     & 1     \\
            \textbf{Con ascensor}                      & 1     & 1     & 1     \\
            \textbf{Tiempo a transporte público (min)} & 5     & 13    & 8     \\
            \hline
        \end{tabular}
        \caption{Listado de propiedades evaluadas - 3}
    \end{table}

    \chapter{Análisis de resultados}
    \label{chap:resultados}

    \chapter{Conclusiones y mejoras}
    \label{chap:conclusion}

    %----------
    %    BIBLIOGRAFÍA
    %----------

    %\nocite{*} % Si quieres que aparezcan en la bibliografía todos los documentos que la componen (también los que no estén citados en el texto) descomenta está lína

    \clearpage

    \phantomsection
    \addcontentsline{toc}{chapter}{Bibliografía}
    \label{chap:bibliography}
    \setquotestyle[english]{british} % Cambiamos el tipo de cita porque en el estilo IEEE se usan las comillas inglesas.
    \printbibliography

    %----------
    %    ANEXOS
    %----------

    % Si tu trabajo incluye anexos, puedes descomentar las siguientes líneas
    %\chapter* {Anexo x}
    %\pagenumbering{gobble} % Las páginas de los anexos no se numeran

\end{document}
